\documentclass[12pt,draft,a4paper]{article}
\usepackage{graphicx}
\usepackage{pslatex}
%\usepackage{german}
\usepackage{amsmath}
\PassOptionsToPackage{righttag}{amsmath}
\usepackage{epsfig,multicol,latexsym,amsfonts,amssymb}
\usepackage{alltt,dcolumn}
%%\usepackage{pwa}

%\usepackage{palatino}

\pagenumbering{arabic}
%%%%%%%%%%%%%%%%%%%%%%%%%%%%%%%%%%%%%%%%%%%%%%%%%%%%
\setlength{\evensidemargin}{-10mm}
\setlength{\oddsidemargin}{0mm}
\setlength{\topmargin}{10mm}
\setlength{\headheight}{0pt}
\setlength{\headsep}{0pt}
\setlength{\marginparsep}{0pt}
\setlength{\marginparwidth}{0pt}
\setlength{\marginparpush}{0pt}
%%%%%%%%%%%%%%%%%%%%%%%%%%%%%%%%%%%%%%%%%%%%%%%%%%%%
%\setlength{\hoffset}{-30mm}
\setlength{\voffset}{-15mm}
\setlength{\textwidth}{170mm}
\setlength{\textheight}{250mm}
\setlength{\footskip}{10mm}
%%%%%%%%%%%%%%%%%%%%%%%%%%%%%%%%%%%%%%%%%%%%%%%%%%%%
\setlength{\parindent}{0em}
%%%%%%%%%%%%%%%%%%%%%%%%%%%%%%%%%%%%%%%%%%%%%%%%%%%%
\setlength{\columnseprule}{0.5pt}
\setlength{\columnsep}{1cm}

%% \input{CHUNG/myown1}

\newcommand{\pim}{\pi^-}
\newcommand{\pipm}{\pi^\mp}
\newcommand{\pit}{\pi^-(1300)}
\newcommand{\pitpm}{\pi^\pm(1300)}
\newcommand{\sg}{\sigma}
\newcommand{\rh}{\rho^0(770)}
\newcommand{\fz}{f_0(1370)}
\newcommand{\fone}{f_1(1285)}
\newcommand{\ftwo}{f_2(1270)}
\newcommand{\aone}{a_1^-(1260)}
\newcommand{\aonepm}{a_1^\pm(1260)}
\newcommand{\atwo}{a_2^-(1320)}
\newcommand{\atwopm}{a_2^\pm(1320)}
\newcommand{\pitwo}{\pi_2^-(1670)}
\newcommand{\pitwopm}{\pi_2^\pm(1670)}
\newcommand{\rhP}{\rho^0({1700 \atop 1450})}

\newcommand{\bu}{$\bullet$}

\begin{document}

\newcommand{\ma}[4]{\begin{pmatrix}
 #1& #2 \\
 #3 & #4
\end{pmatrix}}

%%%%##########################################################################


\section{Optimizing calculations of the Likelihood Function}

In general the Likelihood Function has the following form:

\[L=\sum_i^{\text{Events}}\left[ \log\sum_r^{\text{Ranks}}\left|\sum_\alpha^{\text{Waves}} V_{\alpha}^r A_\alpha^i \right|^2\right]-\text{Normalization}\]

where the $V_\alpha^r$ are the production amplitudes (= fit parameters) in rank $r$ and $A_\alpha^i$ is the (precalculated) decay amplitude of the wave $\alpha$ for event $i$. Both production and decay amplitudes are in general complex valued.

The derivatives with respect to the individual fit parameters are then given by:
\[\frac{dL}{d\Re(V_\alpha)}=\sum_i^{\text{Events}}\left[\frac{2}{L_i}\sum_r^{\text{Ranks}}\sum_{\alpha'}^{\text{Waves}} \Re(V_{\alpha'}^{r} A_{\alpha'}^{i} A_{\alpha}^{i*})  \right] - \frac{d\text{Normalization}}{d\Re(V_\alpha)}\]

where $L_i$ is the contribution to the Likelihood of event $i$ (inside the $log$).For the calculation of the Likelihood the value of $V_{\alpha}^r A_\alpha^i$ is already available at each iteration!


We try to bring this formula into a form that is suitable for effective calculation\footnote{We use the following relations: \[\Re(Z_1Z_2)=\Re((R_1+iI_1)(R_2+iI_2))=\Re(R_1R_2-I_1I_2+i(I_1R_2+R_1I_2))=R_1R_2-I_1I_2\] Applying this to a sum of real parts: \[\sum_j\Re(Z_1Z_2^j)=\sum_jR_1R_2^j-I_1I_2^j=R_1\sum_jR_2^j-I_1\sum_jI_2^j=\Re\left(Z_1\sum_jZ_2^j\right)\]}:

\[\frac{dL}{d\Re(V_\alpha)}=\sum_i^{\text{Events}}\left[\frac{2}{L_i}\sum_r^{\text{Ranks}}\Re\left(A_{\alpha}^{i*}\sum_{\alpha'}^{\text{Waves}} V_{\alpha'}^{r} A_{\alpha'}^{i} \right)  \right] - ...\]


This means roughly a reduction in the number of multiplications corresponding to the number of waves in the fit!


\end{document}
